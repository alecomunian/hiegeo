%% Generated by Sphinx.
\def\sphinxdocclass{report}
\documentclass[letterpaper,10pt,english]{sphinxmanual}
\ifdefined\pdfpxdimen
   \let\sphinxpxdimen\pdfpxdimen\else\newdimen\sphinxpxdimen
\fi \sphinxpxdimen=.75bp\relax

\PassOptionsToPackage{warn}{textcomp}
\usepackage[utf8]{inputenc}
\ifdefined\DeclareUnicodeCharacter
% support both utf8 and utf8x syntaxes
  \ifdefined\DeclareUnicodeCharacterAsOptional
    \def\sphinxDUC#1{\DeclareUnicodeCharacter{"#1}}
  \else
    \let\sphinxDUC\DeclareUnicodeCharacter
  \fi
  \sphinxDUC{00A0}{\nobreakspace}
  \sphinxDUC{2500}{\sphinxunichar{2500}}
  \sphinxDUC{2502}{\sphinxunichar{2502}}
  \sphinxDUC{2514}{\sphinxunichar{2514}}
  \sphinxDUC{251C}{\sphinxunichar{251C}}
  \sphinxDUC{2572}{\textbackslash}
\fi
\usepackage{cmap}
\usepackage[T1]{fontenc}
\usepackage{amsmath,amssymb,amstext}
\usepackage{babel}



\usepackage{times}
\expandafter\ifx\csname T@LGR\endcsname\relax
\else
% LGR was declared as font encoding
  \substitutefont{LGR}{\rmdefault}{cmr}
  \substitutefont{LGR}{\sfdefault}{cmss}
  \substitutefont{LGR}{\ttdefault}{cmtt}
\fi
\expandafter\ifx\csname T@X2\endcsname\relax
  \expandafter\ifx\csname T@T2A\endcsname\relax
  \else
  % T2A was declared as font encoding
    \substitutefont{T2A}{\rmdefault}{cmr}
    \substitutefont{T2A}{\sfdefault}{cmss}
    \substitutefont{T2A}{\ttdefault}{cmtt}
  \fi
\else
% X2 was declared as font encoding
  \substitutefont{X2}{\rmdefault}{cmr}
  \substitutefont{X2}{\sfdefault}{cmss}
  \substitutefont{X2}{\ttdefault}{cmtt}
\fi


\usepackage[Bjarne]{fncychap}
\usepackage{sphinx}

\fvset{fontsize=\small}
\usepackage{geometry}

% Include hyperref last.
\usepackage{hyperref}
% Fix anchor placement for figures with captions.
\usepackage{hypcap}% it must be loaded after hyperref.
% Set up styles of URL: it should be placed after hyperref.
\urlstyle{same}
\addto\captionsenglish{\renewcommand{\contentsname}{Contents:}}

\usepackage{sphinxmessages}
\setcounter{tocdepth}{1}



\title{hiegeo}
\date{Oct 02, 2019}
\release{}
\author{Alessandro Comunian}
\newcommand{\sphinxlogo}{\vbox{}}
\renewcommand{\releasename}{}
\makeindex
\begin{document}

\pagestyle{empty}
\sphinxmaketitle
\pagestyle{plain}
\sphinxtableofcontents
\pagestyle{normal}
\phantomsection\label{\detokenize{index::doc}}


Welcome to the \sphinxtitleref{hiegeo}’s documentation, a Python module to model
geology taking into account for hierarchy and chronological order
among the geological structures.


\chapter{Purpose}
\label{\detokenize{purpose:purpose}}\label{\detokenize{purpose::doc}}
The purpose of the \sphinxtitleref{hiegeo} Python module is to propose a novel
approach to geological modeling, that takes into account for the
hierarchy and the chronological order of the geological structures to
be represented.

See the manuscript \sphinxstyleemphasis{hiegeo: a novel Python module to model
stratigraphic alluvial architectures, constrained by stratigraphic
hierarchy and relative chronology} by Chiara Zuffetti, Alessandro
Comunian, Riccardo Bersezio, and Philippe Renard for more details.

Hereinafter you can see one of the possible output provided by
\sphinxtitleref{hiegeo}, for example a representation of the geology in terms of
stratigraphic boundaries (SBs) including only the 3rd level of hierarchy, the 3rd and the 2nd, and 3rd, 2nd and 1st:

\noindent\sphinxincludegraphics[width=1.000\linewidth]{{SBs_Hierarchy3}.png}

\noindent\sphinxincludegraphics[width=1.000\linewidth]{{SBs_Hierarchy3-2}.png}

\noindent\sphinxincludegraphics[width=1.000\linewidth]{{SBs_Hierarchy3-2-1}.png}

The same representation at different hierarchical levels can be obtained in terms of Stratigraphic Units (SUs):

\noindent\sphinxincludegraphics[width=1.000\linewidth]{{SUs_Hierarchy3}.png}

\noindent\sphinxincludegraphics[width=1.000\linewidth]{{SUs_Hierarchy3-2}.png}

\noindent\sphinxincludegraphics[width=1.000\linewidth]{{SUs_Hierarchy3-2-1}.png}

In addition, you can also have a representation of the geological hierarchy as a tree structure:

\begin{DUlineblock}{0em}
\item[] topo
\item[] S4
\item[] └── S4-1
\item[] S3
\item[] └── S3-1
\item[]
\begin{DUlineblock}{\DUlineblockindent}
\item[] └── S3-1-1
\end{DUlineblock}
\item[] S2
\item[] S1
\item[] └── S1-1
\item[]
\begin{DUlineblock}{\DUlineblockindent}
\item[] └── S1-1-1
\end{DUlineblock}
\item[] S0
\item[] └── S0-1
\end{DUlineblock}


\chapter{Installation}
\label{\detokenize{install:installation}}\label{\detokenize{install::doc}}
Installing and using \sphinxtitleref{hiegeo} should be relatively easy, and thanks to
Python’s flexibility, that should be feasible on many operative
systems (OSs), including MS Windows, Mac OS X and Linux.


\section{Requirements}
\label{\detokenize{install:requirements}}
To use the \sphinxtitleref{hiegeo} module you need a standard Python3.X installation,
together with the main mathematical and plotting libraries (\sphinxcode{\sphinxupquote{numpy}},
\sphinxcode{\sphinxupquote{pandas}}, \sphinxcode{\sphinxupquote{matplotlib}} etc) and the module \sphinxcode{\sphinxupquote{anytree}}, which is
required to handle the hierarchical structures.


\subsection{Python}
\label{\detokenize{install:python}}
Probably, unless you are working with Linux and you can use some
package manager like \sphinxtitleref{Synaptic}, the easy way to have Python up and
running on your system is to use \sphinxtitleref{Anaconda}
(\sphinxurl{https://www.anaconda.com/}). Therefore,
download and install the Python (version 3.X) of Anaconda which is
suitable for your OS by following the instruction provided on the
Anaconda web site.


\subsection{The \sphinxtitleref{anytree} module}
\label{\detokenize{install:the-anytree-module}}
Some capabilities of \sphinxtitleref{hiegeo} are provided by the Python \sphinxtitleref{anytree}
module (\sphinxurl{https://pypi.org/project/anytree/}). Once installed Anaconda, you can open the \sphinxtitleref{Anaconda prompt} and install it by typing:

\begin{sphinxVerbatim}[commandchars=\\\{\}]
\PYG{n}{pip} \PYG{n}{install} \PYG{n}{anytree}
\end{sphinxVerbatim}

To verify if the installation worked properly, you can open a Python
shell and check that the line \sphinxcode{\sphinxupquote{import anytree}} works and does not
provide any error output.


\subsection{The \sphinxtitleref{hiegeo} module}
\label{\detokenize{install:the-hiegeo-module}}
If you keep the provided module file (\sphinxcode{\sphinxupquote{hiegeo.py}}) in the same
directory of the calling script, then everything should work as it is.

Alternatively, if you prefer to have more flexibility, you can put the
file \sphinxcode{\sphinxupquote{hiegeo.py}} in a directory that should be included in your
\sphinxcode{\sphinxupquote{PYTHONPATH}} environmental variable. If you are using the editor
\sphinxtitleref{Spyderlib}, included in Anaconda, to do that you can use the \sphinxcode{\sphinxupquote{Python
Path Manager}}; in that case you will need to \sphinxstylestrong{restart} Spyderlib.


\section{Verification}
\label{\detokenize{install:verification}}
In the end, to verify if \sphinxcode{\sphinxupquote{hiegeo}} was properly installed and
available to your Python script, you can call it from the Python shell with

\begin{sphinxVerbatim}[commandchars=\\\{\}]
\PYG{n}{include} \PYG{n}{hiegeo}
\end{sphinxVerbatim}

to double check is any error rises.


\chapter{Usage}
\label{\detokenize{usage:usage}}\label{\detokenize{usage::doc}}
This section describe an example usage of \sphinxtitleref{hiegeo}.


\section{Input files}
\label{\detokenize{usage:input-files}}
For properly run \sphinxtitleref{hiegeo}, you need two files:
\begin{enumerate}
\def\theenumi{\arabic{enumi}}
\def\labelenumi{\theenumi )}
\makeatletter\def\p@enumii{\p@enumi \theenumi )}\makeatother
\item {} 
A \sphinxtitleref{.JSON} file, that contains info about the discretization grid
and the rendering of the plots.

\item {} 
A data file (\sphinxtitleref{.CSV} format) that include the actual data set, with
information about the contact points, chronology and hierarchy.

\end{enumerate}


\subsection{JSON file}
\label{\detokenize{usage:json-file}}
An example \sphinxtitleref{JSON} file is provided in the file \sphinxcode{\sphinxupquote{hiegeo\_test.json}}. It contains three main sections:
\begin{description}
\item[{colors}] \leavevmode
This section contains the definition (HEX format) of the color that should be given to each SB.
The color should be provided for the highest hierarchy SBs only.

\item[{data\_file}] \leavevmode
The name of the CSV data file containing the dataset itself.

\item[{grid}] \leavevmode
Info about the grid used to discretize the domain. The notation should be straightforward.

\end{description}


\subsection{CSV file}
\label{\detokenize{usage:csv-file}}
This file contains all the required information to create and plot SBs and SUs.

Its header should look like:

\begin{DUlineblock}{0em}
\item[] ,gis\_id,chronology,hierarchy,sb\_name,su\_name,x,z
\item[] 0,1,10,3,S4,LCN,897.1953,134.202585
\item[] 1,2,10,3,S4,LCN,3142.7824,124.47562
\item[] 2,3,10,3,S4,LCN,3797.4044,119.94562
\item[] …
\end{DUlineblock}

Apart from the very first column, that contains some index that are
used by the Python library \sphinxcode{\sphinxupquote{pandas}} (but here they could be set to
whatever), the other columns content should be:
\begin{description}
\item[{gis\_id}] \leavevmode
This is simply a legacy column that contains the ID of the points extracted from the GIS software.
You can set whatever value here as these are not used by \sphinxtitleref{hiegeo}.

\item[{chronology}] \leavevmode
This is a very important column since it contains information about the chronology of the corresponding
SB. The value should be an integer, that grows from the oldest to the youngest SB.

\item[{hierarchy}] \leavevmode
Another very important column, contains the hierarchy (sometimes called “rank”) of the SB. High level of hierarchy
means high importance in sedimentary terms. In the provided example we have integer values ranging from 1 to 3.

\item[{sb\_name}] \leavevmode
This is the name to be given to the SB. It should be a string

\item[{su\_name}] \leavevmode
This is the name of the SU. Actually, this column is not implemented and the SU is named according to
and internally consistent nomenclature.

\item[{x}] \leavevmode
The coordinate along the \(x\) axis of the (contact) points

\item[{y}] \leavevmode
The coordinate along the \(y\) axis of the (contact) points

\end{description}


\section{Example scripts}
\label{\detokenize{usage:example-scripts}}
Here two example scripts are provided. One file
(\sphinxtitleref{hiegeo\_test-simple.py}) contains a full working example to read
data, plot with a basic layout SBs and SUs, and provide a hierarchical
representation of the geology with a tree structure. The other file,
instead (\sphinxtitleref{hiegeo\_test-full.py}) provides a more complete example where
plots are made for three different levels of hierarchical
representation, with advanced plot legend and includes the creation of
a GSLIB output file.

Both script should be sufficiently documented to allow running them
without additional information.


\chapter{Module documentation}
\label{\detokenize{hiegeo:module-hiegeo}}\label{\detokenize{hiegeo:module-documentation}}\label{\detokenize{hiegeo::doc}}\index{hiegeo (module)@\spxentry{hiegeo}\spxextra{module}}\begin{quote}\begin{description}
\item[{This file}] \leavevmode
\sphinxcode{\sphinxupquote{hiegeo.py}}

\item[{Purpose}] \leavevmode
hie-geo: “hierarchical geo-modeling”

A module containing the functions required to read information
from a geological data-set (contact point) and plot in a
hierarchical fashion the unit sections.

\item[{Usage}] \leavevmode
See \sphinxcode{\sphinxupquote{hiegeo\_test-simple.py}} and \sphinxcode{\sphinxupquote{hiegeo\_test-full.py}} for an example usage.

\item[{Version}] \leavevmode\begin{description}
\item[{1.9, 2019-10-02 :}] \leavevmode
Some clean up an adding License information.

\item[{1.8, 2019-07-31 :}] \leavevmode
Version updated with some clean up, before important changes in “rank” 
and “order” definitions…

\end{description}

\item[{Some conventions}] \leavevmode\begin{description}
\item[{SB:}] \leavevmode
Stratigraphic Bound

\item[{SU:}] \leavevmode
Stratigraphic Unit

\item[{Hierarchy:}] \leavevmode
Minumum value stands for lower hierarchical importance;  higher
values are for more important SBs.

\item[{chronology:}] \leavevmode
This is the temporal order of the SBs. Lower values corresponds to older
SBs, while higher values correspond to more recent SBs.

\end{description}

\item[{Authors}] \leavevmode
Alessandro Comunian

\item[{License}] \leavevmode
\sphinxtitleref{hiegeo} is free software: you can redistribute it and/or modify
it under the terms of the GNU General Public License as published by
the Free Software Foundation, either version 3 of the License, or
(at your option) any later version.

\sphinxtitleref{hiegeo} is distributed in the hope that it will be useful,
but WITHOUT ANY WARRANTY; without even the implied warranty of
MERCHANTABILITY or FITNESS FOR A PARTICULAR PURPOSE.  See the
GNU General Public License for more details.

You should have received a copy of the GNU General Public License
along with \sphinxtitleref{hiegeo}.  If not, see \textless{}\sphinxurl{https://www.gnu.org/licenses/}\textgreater{}.

\end{description}\end{quote}

\begin{sphinxadmonition}{note}{Note:}\begin{enumerate}
\def\theenumi{\arabic{enumi}}
\def\labelenumi{\theenumi )}
\makeatletter\def\p@enumii{\p@enumi \theenumi )}\makeatother
\item {} 
At the moment, the points of intersection among surfaces are a
required input data. Also, it is expected that when two SBs are
intersecting at one point, the point is defined in the dataset
for both the SBs.

\end{enumerate}
\end{sphinxadmonition}

ALTRI DUBBI:

“Correlation points” or “contact points”?
\index{SBound (class in hiegeo)@\spxentry{SBound}\spxextra{class in hiegeo}}

\begin{fulllineitems}
\phantomsection\label{\detokenize{hiegeo:hiegeo.SBound}}\pysiglinewithargsret{\sphinxbfcode{\sphinxupquote{class }}\sphinxcode{\sphinxupquote{hiegeo.}}\sphinxbfcode{\sphinxupquote{SBound}}}{\emph{name}, \emph{data=None}, \emph{hierarchy=None}, \emph{chronology=None}, \emph{xd=None}, \emph{color=None}, \emph{parent=None}}{}
A class useful to contain all the properties of a Stratigraphic
Boundary (SB).
\index{broken\_line() (hiegeo.SBound method)@\spxentry{broken\_line()}\spxextra{hiegeo.SBound method}}

\begin{fulllineitems}
\phantomsection\label{\detokenize{hiegeo:hiegeo.SBound.broken_line}}\pysiglinewithargsret{\sphinxbfcode{\sphinxupquote{broken\_line}}}{}{}
This is useful to create, for each point of the provided discretization
along \sphinxstyleemphasis{x} axis, (\sphinxtitleref{self.xd}), a broken line. A \sphinxtitleref{numpy.nan} value is set
where the provided “raw” coordinates does not allow to complete the 
discretization.

\end{fulllineitems}

\index{get\_ancestors() (hiegeo.SBound method)@\spxentry{get\_ancestors()}\spxextra{hiegeo.SBound method}}

\begin{fulllineitems}
\phantomsection\label{\detokenize{hiegeo:hiegeo.SBound.get_ancestors}}\pysiglinewithargsret{\sphinxbfcode{\sphinxupquote{get\_ancestors}}}{\emph{anc\_list={[}{]}}}{}
Provide a list of ancestors of the current SB.
\begin{description}
\item[{Parameters:}] \leavevmode\begin{description}
\item[{anc\_list: list}] \leavevmode
A partially filled input list can be provided, to
allow appending additional ancestors.

\end{description}

\end{description}

\end{fulllineitems}

\index{get\_obj\_above() (hiegeo.SBound method)@\spxentry{get\_obj\_above()}\spxextra{hiegeo.SBound method}}

\begin{fulllineitems}
\phantomsection\label{\detokenize{hiegeo:hiegeo.SBound.get_obj_above}}\pysiglinewithargsret{\sphinxbfcode{\sphinxupquote{get\_obj\_above}}}{\emph{hierarchy}}{}
Get all the SBs defined in the script with the given hierarchy,
ordeder by “chronology”.
\begin{description}
\item[{Parameters:}] \leavevmode\begin{description}
\item[{hierarchy: integer}] \leavevmode
The hierarchy of the SBs to be plotted.

\end{description}

\item[{Returns:}] \leavevmode
A list containing all the defined SBs with the given hierarchy, 
ordered by “chronology”.

\end{description}

\end{fulllineitems}

\index{plot() (hiegeo.SBound method)@\spxentry{plot()}\spxextra{hiegeo.SBound method}}

\begin{fulllineitems}
\phantomsection\label{\detokenize{hiegeo:hiegeo.SBound.plot}}\pysiglinewithargsret{\sphinxbfcode{\sphinxupquote{plot}}}{\emph{lw=None}, \emph{chronology=None}}{}
Plot the SB.
\begin{description}
\item[{Parameters:}] \leavevmode\begin{description}
\item[{lw: integer}] \leavevmode
This is the line width

\item[{chronology: integer}] \leavevmode
The relative chronology of the SB.

\end{description}

\end{description}

\begin{sphinxadmonition}{note}{Note:}
The values where \sphinxcode{\sphinxupquote{zd==np.nan}} are not plotted.
\end{sphinxadmonition}

\end{fulllineitems}

\index{plot\_ax() (hiegeo.SBound method)@\spxentry{plot\_ax()}\spxextra{hiegeo.SBound method}}

\begin{fulllineitems}
\phantomsection\label{\detokenize{hiegeo:hiegeo.SBound.plot_ax}}\pysiglinewithargsret{\sphinxbfcode{\sphinxupquote{plot\_ax}}}{\emph{ax}, \emph{lw=None}, \emph{color=None}}{}
Plot the SB when a matplotlib axis is provided.
\begin{description}
\item[{Parameters:}] \leavevmode\begin{description}
\item[{lw: integer (optional)}] \leavevmode
The line width. Using directly the hierarchies (for example,
with hierarchies from 1 to 3) seems to provide a figure which
is OK

\item[{color: color code (optional)}] \leavevmode
The line color.

\end{description}

\end{description}

\begin{sphinxadmonition}{note}{Note:}
The values where \sphinxtitleref{zd==np.nan} are not plotted.
\end{sphinxadmonition}

\end{fulllineitems}

\index{plot\_fill() (hiegeo.SBound method)@\spxentry{plot\_fill()}\spxextra{hiegeo.SBound method}}

\begin{fulllineitems}
\phantomsection\label{\detokenize{hiegeo:hiegeo.SBound.plot_fill}}\pysiglinewithargsret{\sphinxbfcode{\sphinxupquote{plot\_fill}}}{\emph{ax}, \emph{hierarchy}, \emph{alpha}}{}
“fill\_between” plot of a SU given the bottom SB.
\begin{description}
\item[{Parameters:}] \leavevmode\begin{description}
\item[{ax: matplotlib axis}] \leavevmode
The object where to make the plot

\item[{hierarchy: integer}] \leavevmode
Hierarchy.

\item[{alpha: float}] \leavevmode
Transparency level.

\end{description}

\end{description}

\begin{sphinxadmonition}{note}{Note:}
Using \sphinxtitleref{alpha} as parameter could create problems when one 
tries to properly represent overlapping regions.
(and this, in principle, should happen quite frequently when
dealing with hierarchical models…)
\end{sphinxadmonition}

\end{fulllineitems}

\index{print\_hie() (hiegeo.SBound method)@\spxentry{print\_hie()}\spxextra{hiegeo.SBound method}}

\begin{fulllineitems}
\phantomsection\label{\detokenize{hiegeo:hiegeo.SBound.print_hie}}\pysiglinewithargsret{\sphinxbfcode{\sphinxupquote{print\_hie}}}{}{}
Print out some info about a SB, in a hierarchical fashion.

\end{fulllineitems}


\end{fulllineitems}

\index{SUnit (class in hiegeo)@\spxentry{SUnit}\spxextra{class in hiegeo}}

\begin{fulllineitems}
\phantomsection\label{\detokenize{hiegeo:hiegeo.SUnit}}\pysiglinewithargsret{\sphinxbfcode{\sphinxupquote{class }}\sphinxcode{\sphinxupquote{hiegeo.}}\sphinxbfcode{\sphinxupquote{SUnit}}}{\emph{name=None}, \emph{bot=None}, \emph{top=None}, \emph{sbs=None}, \emph{id=None}}{}
A class useful to contain all the properties related to a SU.

\begin{sphinxadmonition}{note}{Note:}
Is is expected that in the same script all the SBounds were already 
created extracting the information from the correlation/contact points file.
\end{sphinxadmonition}

\begin{sphinxadmonition}{warning}{Warning:}
The development of this Class is still work in progress, and only
basic functionalities are provided for the moment.
\end{sphinxadmonition}
\index{plot() (hiegeo.SUnit method)@\spxentry{plot()}\spxextra{hiegeo.SUnit method}}

\begin{fulllineitems}
\phantomsection\label{\detokenize{hiegeo:hiegeo.SUnit.plot}}\pysiglinewithargsret{\sphinxbfcode{\sphinxupquote{plot}}}{}{}
Plot a SU

\end{fulllineitems}


\end{fulllineitems}

\index{check\_data\_ingrid() (in module hiegeo)@\spxentry{check\_data\_ingrid()}\spxextra{in module hiegeo}}

\begin{fulllineitems}
\phantomsection\label{\detokenize{hiegeo:hiegeo.check_data_ingrid}}\pysiglinewithargsret{\sphinxcode{\sphinxupquote{hiegeo.}}\sphinxbfcode{\sphinxupquote{check\_data\_ingrid}}}{\emph{data}, \emph{x}, \emph{z}}{}
Check if data are contained in the discretization grid.
\begin{description}
\item[{Parameters:}] \leavevmode\begin{description}
\item[{data: (pandas DataFrame)}] \leavevmode
The input data set. See the documentation for more details.

\item[{x, y: numpy arrays}] \leavevmode
These are the discretization grid coordinates.

\end{description}

\end{description}

\begin{sphinxadmonition}{note}{Note:}
Actually, the warning message in general is not a “nasty” one, since it only
states that the definining points of some SBs were not available on all the
vertical coordinates, and therefore the domain was restricted.
\end{sphinxadmonition}

\end{fulllineitems}

\index{create\_sb\_by\_hierarchy() (in module hiegeo)@\spxentry{create\_sb\_by\_hierarchy()}\spxextra{in module hiegeo}}

\begin{fulllineitems}
\phantomsection\label{\detokenize{hiegeo:hiegeo.create_sb_by_hierarchy}}\pysiglinewithargsret{\sphinxcode{\sphinxupquote{hiegeo.}}\sphinxbfcode{\sphinxupquote{create\_sb\_by\_hierarchy}}}{\emph{data}, \emph{hierarchy}, \emph{xd}, \emph{color=None}}{}
Given a data set containing the correlation points and a hierarchy,
create a list of SB objects.
\begin{description}
\item[{Parameters:}] \leavevmode\begin{description}
\item[{data: DataFrame}] \leavevmode
The info about the correlation points

\item[{hierarchy: integer}] \leavevmode
The hierarchy that should be created

\item[{xd: numpy array}] \leavevmode
Discretized \sphinxstyleemphasis{x} coordinate

\end{description}

\item[{Returns:}] \leavevmode
A list containing the SB objects, ordered by “chronology”

\end{description}

\end{fulllineitems}

\index{create\_sb\_from\_data() (in module hiegeo)@\spxentry{create\_sb\_from\_data()}\spxextra{in module hiegeo}}

\begin{fulllineitems}
\phantomsection\label{\detokenize{hiegeo:hiegeo.create_sb_from_data}}\pysiglinewithargsret{\sphinxcode{\sphinxupquote{hiegeo.}}\sphinxbfcode{\sphinxupquote{create\_sb\_from\_data}}}{\emph{data}, \emph{xd}, \emph{color=None}}{}
Create all the SBs defined in a dataset
\begin{description}
\item[{Parameters:}] \leavevmode\begin{description}
\item[{data: DataFrame}] \leavevmode
Info about the correlation points

\end{description}

\item[{Returns:}] \leavevmode
A dictionary containing, for each hierarchy, a list
of SBs.

\end{description}

\end{fulllineitems}

\index{get\_SBparent\_by\_name() (in module hiegeo)@\spxentry{get\_SBparent\_by\_name()}\spxextra{in module hiegeo}}

\begin{fulllineitems}
\phantomsection\label{\detokenize{hiegeo:hiegeo.get_SBparent_by_name}}\pysiglinewithargsret{\sphinxcode{\sphinxupquote{hiegeo.}}\sphinxbfcode{\sphinxupquote{get\_SBparent\_by\_name}}}{\emph{data}, \emph{sb\_name}}{}
Given a data set, find the “parent” SB of a SB with a given name.
\begin{description}
\item[{Parameters:}] \leavevmode\begin{description}
\item[{data: data set (see package documentation for details)}] \leavevmode
The data set where to look for parents.

\item[{sb\_name: string}] \leavevmode
The name of the SB of which looking for a “parent”

\end{description}

\item[{Return:}] \leavevmode
The parent object SB

\end{description}

\end{fulllineitems}

\index{get\_SU\_by\_bot() (in module hiegeo)@\spxentry{get\_SU\_by\_bot()}\spxextra{in module hiegeo}}

\begin{fulllineitems}
\phantomsection\label{\detokenize{hiegeo:hiegeo.get_SU_by_bot}}\pysiglinewithargsret{\sphinxcode{\sphinxupquote{hiegeo.}}\sphinxbfcode{\sphinxupquote{get\_SU\_by\_bot}}}{\emph{bot}}{}
Find the SU object corresponding to a given bottom SB.

\end{fulllineitems}

\index{get\_allmin() (in module hiegeo)@\spxentry{get\_allmin()}\spxextra{in module hiegeo}}

\begin{fulllineitems}
\phantomsection\label{\detokenize{hiegeo:hiegeo.get_allmin}}\pysiglinewithargsret{\sphinxcode{\sphinxupquote{hiegeo.}}\sphinxbfcode{\sphinxupquote{get\_allmin}}}{\emph{sbs}}{}
Given a list of objects, return the min of the \sphinxstyleemphasis{z} coordinate.

\begin{sphinxadmonition}{note}{Note:}
Warnings about all columns containing a NaN in \sphinxtitleref{nanmin} is suppressed
as it should not be harmful.
\end{sphinxadmonition}

\end{fulllineitems}

\index{get\_bot\_sb() (in module hiegeo)@\spxentry{get\_bot\_sb()}\spxextra{in module hiegeo}}

\begin{fulllineitems}
\phantomsection\label{\detokenize{hiegeo:hiegeo.get_bot_sb}}\pysiglinewithargsret{\sphinxcode{\sphinxupquote{hiegeo.}}\sphinxbfcode{\sphinxupquote{get\_bot\_sb}}}{\emph{sb}, \emph{sbs}}{}
Get a list of SB objects that are potential candidates to
provide the bottom for the given SB.
\begin{description}
\item[{Parameters:}] \leavevmode
sb: SBound
sbs: list of SBound objects

\end{description}

\end{fulllineitems}

\index{get\_intersect() (in module hiegeo)@\spxentry{get\_intersect()}\spxextra{in module hiegeo}}

\begin{fulllineitems}
\phantomsection\label{\detokenize{hiegeo:hiegeo.get_intersect}}\pysiglinewithargsret{\sphinxcode{\sphinxupquote{hiegeo.}}\sphinxbfcode{\sphinxupquote{get\_intersect}}}{\emph{name}, \emph{data}}{}
Get a dataframe of points of SBs that intersect with the SB “name”.
If there is no intersection, the output dataframe will be empty.
\begin{description}
\item[{Parameters:}] \leavevmode\begin{description}
\item[{name: }] \leavevmode
Name of the current SB we are working with

\item[{data:}] \leavevmode
Dataset containing the point correlations

\end{description}

\item[{Returns:}] \leavevmode
A daframe containing the intersection points.

\end{description}

\end{fulllineitems}

\index{get\_max\_chronology() (in module hiegeo)@\spxentry{get\_max\_chronology()}\spxextra{in module hiegeo}}

\begin{fulllineitems}
\phantomsection\label{\detokenize{hiegeo:hiegeo.get_max_chronology}}\pysiglinewithargsret{\sphinxcode{\sphinxupquote{hiegeo.}}\sphinxbfcode{\sphinxupquote{get\_max\_chronology}}}{\emph{data}}{}
Get the maximum chronology defined in the dataset.
\begin{description}
\item[{Parameters:}] \leavevmode\begin{description}
\item[{data: DataFrame}] \leavevmode
Info about the correlation points

\end{description}

\item[{Return:}] \leavevmode
The value of the max chronology

\end{description}

\end{fulllineitems}

\index{get\_newer\_sb\_same\_hie() (in module hiegeo)@\spxentry{get\_newer\_sb\_same\_hie()}\spxextra{in module hiegeo}}

\begin{fulllineitems}
\phantomsection\label{\detokenize{hiegeo:hiegeo.get_newer_sb_same_hie}}\pysiglinewithargsret{\sphinxcode{\sphinxupquote{hiegeo.}}\sphinxbfcode{\sphinxupquote{get\_newer\_sb\_same\_hie}}}{\emph{sb}}{}
Get all the defined SBs with the same hierarchy but
of greater chronology.
\begin{description}
\item[{Parameters:}] \leavevmode\begin{description}
\item[{sb: SBound}] \leavevmode
The interested SB.

\end{description}

\end{description}

\end{fulllineitems}

\index{get\_parent\_by\_name() (in module hiegeo)@\spxentry{get\_parent\_by\_name()}\spxextra{in module hiegeo}}

\begin{fulllineitems}
\phantomsection\label{\detokenize{hiegeo:hiegeo.get_parent_by_name}}\pysiglinewithargsret{\sphinxcode{\sphinxupquote{hiegeo.}}\sphinxbfcode{\sphinxupquote{get\_parent\_by\_name}}}{\emph{data}, \emph{sb\_name}}{}
Given a data set, find the “parent” SB of a SB with a given name.
\begin{description}
\item[{Parameters:}] \leavevmode\begin{description}
\item[{data: data set (see package documentation for details)}] \leavevmode
The data set where to look for parents.

\item[{sb\_name: string}] \leavevmode
The name of the SB of which looking for a “parent”

\end{description}

\item[{Return:}] \leavevmode
Name of the parent SB

\end{description}

\end{fulllineitems}

\index{get\_sb\_chronology() (in module hiegeo)@\spxentry{get\_sb\_chronology()}\spxextra{in module hiegeo}}

\begin{fulllineitems}
\phantomsection\label{\detokenize{hiegeo:hiegeo.get_sb_chronology}}\pysiglinewithargsret{\sphinxcode{\sphinxupquote{hiegeo.}}\sphinxbfcode{\sphinxupquote{get\_sb\_chronology}}}{\emph{data}, \emph{sb}}{}
Get the chronology of the current SB.
\begin{description}
\item[{Parameters:}] \leavevmode\begin{description}
\item[{data: DataFrame}] \leavevmode
The point correlation info contained in the DBF file

\item[{sb: string}] \leavevmode
Name of the SB.

\end{description}

\end{description}

\end{fulllineitems}

\index{get\_sb\_hierarchy() (in module hiegeo)@\spxentry{get\_sb\_hierarchy()}\spxextra{in module hiegeo}}

\begin{fulllineitems}
\phantomsection\label{\detokenize{hiegeo:hiegeo.get_sb_hierarchy}}\pysiglinewithargsret{\sphinxcode{\sphinxupquote{hiegeo.}}\sphinxbfcode{\sphinxupquote{get\_sb\_hierarchy}}}{\emph{data}, \emph{sb}}{}
Get the hierarchy of the current SB.
\begin{description}
\item[{Parameters:}] \leavevmode\begin{description}
\item[{data: DataFrame}] \leavevmode
The point correlation info contained in the DBF file

\item[{sb: string}] \leavevmode
Name of the SB.

\end{description}

\end{description}

\end{fulllineitems}

\index{get\_sbobj() (in module hiegeo)@\spxentry{get\_sbobj()}\spxextra{in module hiegeo}}

\begin{fulllineitems}
\phantomsection\label{\detokenize{hiegeo:hiegeo.get_sbobj}}\pysiglinewithargsret{\sphinxcode{\sphinxupquote{hiegeo.}}\sphinxbfcode{\sphinxupquote{get\_sbobj}}}{}{}
Get all the SBs defined within the script,
ordeder by “chronology”.
\begin{description}
\item[{Returns:}] \leavevmode
A list containing all the defined SBs ordered by “chronology”.

\end{description}

\end{fulllineitems}

\index{get\_sbobj\_by\_hie() (in module hiegeo)@\spxentry{get\_sbobj\_by\_hie()}\spxextra{in module hiegeo}}

\begin{fulllineitems}
\phantomsection\label{\detokenize{hiegeo:hiegeo.get_sbobj_by_hie}}\pysiglinewithargsret{\sphinxcode{\sphinxupquote{hiegeo.}}\sphinxbfcode{\sphinxupquote{get\_sbobj\_by\_hie}}}{\emph{hierarchy}, \emph{mode='eq'}, \emph{reverse=False}}{}
Get all the SBs defined in the script with the given hierarchy,
ordeder by “chronology”.
\begin{description}
\item[{Parameters:}] \leavevmode\begin{description}
\item[{hierarchy: integer}] \leavevmode
The hierarchy of the SBs to be plotted.

\item[{mode: in (“eq” or “ge”)}] \leavevmode
Depending on this optional value, one can get all the SBs
of the same hierarchy (mode=”eq”) or all the objects with
a \textgreater{}= hierarchy.

\end{description}

\item[{Returns:}] \leavevmode
A list containing all the defined SBs with the given hierarchy, 
ordered by “chronology”.

\end{description}

\end{fulllineitems}

\index{get\_sbobj\_by\_name() (in module hiegeo)@\spxentry{get\_sbobj\_by\_name()}\spxextra{in module hiegeo}}

\begin{fulllineitems}
\phantomsection\label{\detokenize{hiegeo:hiegeo.get_sbobj_by_name}}\pysiglinewithargsret{\sphinxcode{\sphinxupquote{hiegeo.}}\sphinxbfcode{\sphinxupquote{get\_sbobj\_by\_name}}}{\emph{sb\_name}}{}
Get all the SBs defined within the script,
ordeder by “chronology”.
\begin{description}
\item[{Returns:}] \leavevmode
A list containing all the defined SBs ordered by “chronology”.

\end{description}

\end{fulllineitems}

\index{get\_top\_sb() (in module hiegeo)@\spxentry{get\_top\_sb()}\spxextra{in module hiegeo}}

\begin{fulllineitems}
\phantomsection\label{\detokenize{hiegeo:hiegeo.get_top_sb}}\pysiglinewithargsret{\sphinxcode{\sphinxupquote{hiegeo.}}\sphinxbfcode{\sphinxupquote{get\_top\_sb}}}{\emph{sb}, \emph{sbs}}{}
Get a list of SB objects that are potential candidates to
provide the bottom for the given SB
\begin{description}
\item[{Parameters:}] \leavevmode
sb: SBound
sbs: list of SBound objects

\end{description}

\end{fulllineitems}

\index{get\_unique\_chronology() (in module hiegeo)@\spxentry{get\_unique\_chronology()}\spxextra{in module hiegeo}}

\begin{fulllineitems}
\phantomsection\label{\detokenize{hiegeo:hiegeo.get_unique_chronology}}\pysiglinewithargsret{\sphinxcode{\sphinxupquote{hiegeo.}}\sphinxbfcode{\sphinxupquote{get\_unique\_chronology}}}{\emph{data}}{}
Extract from a dataset al the defined chronologies.
\begin{description}
\item[{Parameters:}] \leavevmode\begin{description}
\item[{data: DataFrame}] \leavevmode
Info about the correlation points

\end{description}

\item[{Return:}] \leavevmode
A list containing the defined chronologies.

\end{description}

\end{fulllineitems}

\index{get\_unique\_hierarchy() (in module hiegeo)@\spxentry{get\_unique\_hierarchy()}\spxextra{in module hiegeo}}

\begin{fulllineitems}
\phantomsection\label{\detokenize{hiegeo:hiegeo.get_unique_hierarchy}}\pysiglinewithargsret{\sphinxcode{\sphinxupquote{hiegeo.}}\sphinxbfcode{\sphinxupquote{get\_unique\_hierarchy}}}{\emph{data}}{}
Extract from a dataset al the defined hierarchies.
\begin{description}
\item[{Parameters:}] \leavevmode\begin{description}
\item[{data: DataFrame}] \leavevmode
Info about the correlation points

\end{description}

\item[{Return:}] \leavevmode
A list containing the defined hierarchies.

\end{description}

\end{fulllineitems}

\index{get\_unique\_sb\_name() (in module hiegeo)@\spxentry{get\_unique\_sb\_name()}\spxextra{in module hiegeo}}

\begin{fulllineitems}
\phantomsection\label{\detokenize{hiegeo:hiegeo.get_unique_sb_name}}\pysiglinewithargsret{\sphinxcode{\sphinxupquote{hiegeo.}}\sphinxbfcode{\sphinxupquote{get\_unique\_sb\_name}}}{\emph{data}}{}
Get an ordered list containing the unique names of the 
SBs.
\begin{description}
\item[{Parameters:}] \leavevmode\begin{description}
\item[{data: input DataFrame}] \leavevmode
See the documentation for more details about the data format.

\end{description}

\item[{Returns:}] \leavevmode
A list containing the ordered ids

\end{description}

\end{fulllineitems}

\index{migrate() (in module hiegeo)@\spxentry{migrate()}\spxextra{in module hiegeo}}

\begin{fulllineitems}
\phantomsection\label{\detokenize{hiegeo:hiegeo.migrate}}\pysiglinewithargsret{\sphinxcode{\sphinxupquote{hiegeo.}}\sphinxbfcode{\sphinxupquote{migrate}}}{\emph{df}, \emph{xmesh}, \emph{zmesh}}{}
Migrate the “sparse” coordinates of some points contained in the “x” and
“z” of a DataFrame into the points or a regularly spaced grid.
\begin{description}
\item[{Parameters:}] \leavevmode\begin{description}
\item[{df: pandas DataFrame}] \leavevmode
The dataframe containing the coordinates “x” and “z” to be migrated.

\item[{xmesh: “x” output from meshgrid}] \leavevmode
A matrix containing the \sphinxstyleemphasis{x} coordinates of a structured grid.

\item[{zmesh: “z” output from meshgrid}] \leavevmode
A matrix containing the \sphinxstyleemphasis{z} coordinates of a structured grid.

\end{description}

\end{description}

\end{fulllineitems}

\index{plot\_sb\_ax() (in module hiegeo)@\spxentry{plot\_sb\_ax()}\spxextra{in module hiegeo}}

\begin{fulllineitems}
\phantomsection\label{\detokenize{hiegeo:hiegeo.plot_sb_ax}}\pysiglinewithargsret{\sphinxcode{\sphinxupquote{hiegeo.}}\sphinxbfcode{\sphinxupquote{plot\_sb\_ax}}}{\emph{ax}, \emph{hies}, \emph{palette=None}}{}
Plot all the defined objects for all the provided hierarchies.
\begin{description}
\item[{Parameters:}] \leavevmode\begin{description}
\item[{ax:}] \leavevmode
The axis object

\item[{hies: List of integers.}] \leavevmode
Hierarchies to be plotted.

\item[{palette: Dictionary}] \leavevmode
A dictionary containing the codes corresponding to each SB.

\end{description}

\end{description}

\end{fulllineitems}

\index{plot\_sb\_by\_hie() (in module hiegeo)@\spxentry{plot\_sb\_by\_hie()}\spxextra{in module hiegeo}}

\begin{fulllineitems}
\phantomsection\label{\detokenize{hiegeo:hiegeo.plot_sb_by_hie}}\pysiglinewithargsret{\sphinxcode{\sphinxupquote{hiegeo.}}\sphinxbfcode{\sphinxupquote{plot\_sb\_by\_hie}}}{\emph{hierarchy}}{}
Plot all the SB of a given hierarchy.
\begin{description}
\item[{Parameters:}] \leavevmode\begin{description}
\item[{hierarchy: integer}] \leavevmode
The hierarchy of the SBs to be plotted

\end{description}

\end{description}

\end{fulllineitems}

\index{plot\_sb\_by\_hie\_ax() (in module hiegeo)@\spxentry{plot\_sb\_by\_hie\_ax()}\spxextra{in module hiegeo}}

\begin{fulllineitems}
\phantomsection\label{\detokenize{hiegeo:hiegeo.plot_sb_by_hie_ax}}\pysiglinewithargsret{\sphinxcode{\sphinxupquote{hiegeo.}}\sphinxbfcode{\sphinxupquote{plot\_sb\_by\_hie\_ax}}}{\emph{ax}, \emph{hierarchy}, \emph{palette=None}}{}
Plot all the SB of a given hierarchy.
\begin{description}
\item[{Parameters:}] \leavevmode\begin{description}
\item[{ax: axis object}] \leavevmode
The axis object where the plot should be made

\item[{hierarchy: integer}] \leavevmode
The hierarchy of the SBs to be plotted

\item[{palette: dictionary}] \leavevmode
A dictionary containing the correspondence between
SB name and plotting color.

\end{description}

\end{description}

\end{fulllineitems}

\index{plot\_sb\_fig() (in module hiegeo)@\spxentry{plot\_sb\_fig()}\spxextra{in module hiegeo}}

\begin{fulllineitems}
\phantomsection\label{\detokenize{hiegeo:hiegeo.plot_sb_fig}}\pysiglinewithargsret{\sphinxcode{\sphinxupquote{hiegeo.}}\sphinxbfcode{\sphinxupquote{plot\_sb\_fig}}}{\emph{fig}, \emph{hies}, \emph{palette=None}}{}
Plot all the defined objects for all the provided hierarchies.
\begin{description}
\item[{Parameters:}] \leavevmode\begin{description}
\item[{fig:}] \leavevmode
The axis object

\item[{hies: List of integers.}] \leavevmode
Hierarchies to be plotted.

\item[{palette: Dictionary}] \leavevmode
A dictionary containing the codes corresponding to each SB.

\end{description}

\end{description}

\end{fulllineitems}

\index{plot\_su\_ax() (in module hiegeo)@\spxentry{plot\_su\_ax()}\spxextra{in module hiegeo}}

\begin{fulllineitems}
\phantomsection\label{\detokenize{hiegeo:hiegeo.plot_su_ax}}\pysiglinewithargsret{\sphinxcode{\sphinxupquote{hiegeo.}}\sphinxbfcode{\sphinxupquote{plot\_su\_ax}}}{\emph{ax}, \emph{hierarchy}, \emph{hie\_alpha}}{}
Plot a SU provided a matplotlib axis.
\begin{description}
\item[{Parameters:}] \leavevmode\begin{description}
\item[{ax: matplotlib axis}] \leavevmode
The axis where to plot the SU.

\item[{hierarchy: integer}] \leavevmode
The min hierarchy to be plotted.

\item[{hie\_alpha: float}] \leavevmode
Transparency value

\end{description}

\end{description}

\end{fulllineitems}

\index{print\_data\_info() (in module hiegeo)@\spxentry{print\_data\_info()}\spxextra{in module hiegeo}}

\begin{fulllineitems}
\phantomsection\label{\detokenize{hiegeo:hiegeo.print_data_info}}\pysiglinewithargsret{\sphinxcode{\sphinxupquote{hiegeo.}}\sphinxbfcode{\sphinxupquote{print\_data\_info}}}{\emph{data}}{}
Print some info about the data.

\end{fulllineitems}

\index{print\_start() (in module hiegeo)@\spxentry{print\_start()}\spxextra{in module hiegeo}}

\begin{fulllineitems}
\phantomsection\label{\detokenize{hiegeo:hiegeo.print_start}}\pysiglinewithargsret{\sphinxcode{\sphinxupquote{hiegeo.}}\sphinxbfcode{\sphinxupquote{print\_start}}}{}{}
Print a header with the module name and the version.

\end{fulllineitems}



\chapter{Notation}
\label{\detokenize{index:notation}}
Hereinafter some acronyms that will be used throughout the document:
\begin{description}
\item[{SB}] \leavevmode
Stratigraphic boundaries.

\item[{SU}] \leavevmode
Stratigraphic units.

\end{description}


\chapter{Indices and tables}
\label{\detokenize{index:indices-and-tables}}\begin{itemize}
\item {} 
\DUrole{xref,std,std-ref}{genindex}

\end{itemize}


\renewcommand{\indexname}{Python Module Index}
\begin{sphinxtheindex}
\let\bigletter\sphinxstyleindexlettergroup
\bigletter{h}
\item\relax\sphinxstyleindexentry{hiegeo}\sphinxstyleindexpageref{hiegeo:\detokenize{module-hiegeo}}
\end{sphinxtheindex}

\renewcommand{\indexname}{Index}
\printindex
\end{document}